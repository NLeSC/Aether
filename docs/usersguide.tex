\documentclass[a4paper,10pt]{article}

%\usepackage{graphicx}
\usepackage{url}
\usepackage{times}

\begin{document}

\title{Ibis Communication Library User's Guide}

\author{http://www.cs.vu.nl/ibis}

\maketitle

\section{Introduction}

This manual describes the steps required to run an application that 
uses the Ibis communication library. How to create such an application
is described in the IPL Programmers manual.

A central concept in Ibis is the \emph{Pool}. A pool consists of one or
more Ibis instances, usually running on different machines. Each pool is
generally made up of Ibises running a single distributed application.
Ibises in a pool can communicate with each other, and, using the
registry mechanism present in Ibis, can search for other Ibises in the
same pool, get notified of Ibises joining the pool, etc. To
coordinate Ibis pools a so-called \emph{Ibis server} is used.

\section{The Ibis Server}

The Ibis server is the Swiss-army-knife server of the Ibis project.
Services can be dynamically added to the server. By default, the Ibis
communication library comes with a registry service. This registry
service manages pools, possibly multiple pools at the same time.  

In addition to the registry service, the server also allows
Ibises to route traffic over the server if no direct connection is
possible between two instances due to firewalls or NAT boxes. This is
done using the Smartsockets library of the Ibis project.

The Ibis server is started with the \texttt{ipl-server} script which is
located in the \texttt{scripts} directory of the Ibis distribution.  Before
starting an Ibis application, an Ibis server needs to be running on a
machine that is accessible from all nodes participating in the Ibis run.
The server listens to a TCP port. The port number can be specified using
the \texttt{--port} command line option to the \texttt{ipl-server}
script.  For a complete list of all options, use the \texttt{--help}
option of the script. One useful option is the  \texttt{--events}
option, which makes the registry print out events (such as Ibises
joining a pool).

\subsection{Hubs}
\label{hubs}

The Ibis server is a single point which needs to be reachable from every
Ibis instance. Since sometimes this is not possible due to firewalls,
additional \emph{hubs} can be started to route traffic, creating a
routing infrastructure for the Ibis instances. These hubs can be started
by using ipl-server script with the \texttt{--hub-only} option. In
addition, each hub needs to know the location of as many of the other
hubs as possible. This information can be provided by using the
\texttt{--hub-addresses} option. See the \texttt{--help} option of the
ipl-server script for more information.

\section{Running an Ibis Application}

When the Ibis server is running, the application itself can be
started. There are a number of requirements that need to be met before 
Ibis can be started correctly. In this section we will discuss these in detail.

Several of the steps below require the usage of \emph{system properties}. 
System properties can be set in Java using the \texttt{-D} option of the 
\texttt{java} command. Be sure to use appropriate quoting for your
command interpreter.

As an alternative to using system properties, it is also possible to use
a java properties file
\footnote{\url{http://download.oracle.com/javase/1.5.0/docs/api/java/util/Properties.html}}.
A properties file is a file containing one property per line, usually of
the format \texttt{property = value}. Properties of Ibis can be set in
such a file as if they were set on the command line directly.

Ibis will look for a file named \texttt{ibis.properties} in the current working
directory, on the class path, and at a location specified with the
\texttt{ibis.properties.file} system property.

\subsection{Add ipl.jar to the class path}

An application interfaces to Ibis using the \emph{Ibis Portability Layer} (IPL).
The code for this package is provided in a single jar file:
ipl.jar, appended with the version of ibis, for instance \texttt{ipl-2.3.jar}.
It lives in the \emph{lib} directory of the Ibis distribution.
This jar file needs to be added to the class path of the application.

\subsection{Provide the Ibis implementations}

The IPL loads the actual Ibis implementation dynamically. 
These implementations (and their dependencies) can be provided in two 
ways:

\begin{enumerate} 
\item Add the jar files of the implementations and their dependencies to
      the class path 
\item Set the \texttt{ibis.implementation.path} system property to the
      location of the Ibis implementations and dependencies.  
\end{enumerate}

The \texttt{ibis.implementation.path} property is a list of directories,
separated by the default path separator of your operating system. In
Unix, this is the \texttt{:} character, in Windows it is a \texttt{;}.

\subsection{Logging}

By default, Ibis uses the Log4J library of the Apache project to print
debugging information, warnings, and error messages. This library must be
initialized. A configuration file can be specified using the
\texttt{log4j.configuration} system property. For example, to use a file
named \texttt{log4j.properties} in the current directory, use the
following command line option:
\texttt{-Dlog4j.configuration=file:log4j.properties} . For more info,
see the log4j website \footnote{\url{http://logging.apache.org/log4j}}.

It is possible to use other logging systems, since Ibis in fact uses
Slf4J, which supports various logging systems. In order to use another
logging system, the corresponding jars (both the logging system and the
mapping from slf4j to that logging system) need to be specified on the
classpath.

\subsection{Set the location of the server and hubs}

To communicate with the registry service, each Ibis instance needs the address 
of the Ibis server. This address must be specified by using the 
\texttt{ibis.server.address} system property. The full address needed is 
printed on start up of the Ibis server. 

For convenience, it is also possible to only provide a hostname, port number 
pair, e.g. \texttt{machine.domain.com:5435} or even simply a host, e.g. 
\texttt{localhost}. In this case, the default port number (8888) is implied. 
The port number provided must match the one given to the Ibis server
with the \texttt{--port} option.

When additional hubs are started (see Section \ref{hubs}), their locations 
must be provided to the Ibis instances. This can be done using 
the \texttt{ibis.hub.addresses} property. Ibis expects a comma-separated
list of addresses of hubs. Ibis will use the first reachable hub on the
list. The address of the Ibis server is appended to this list
automatically. Thus, by default, the Ibis server itself is used as the
hub.

\subsection{Set the name and size of the pool}

Each Ibis instance belongs to a pool. The name of this pool must be provided 
using the \texttt{ibis.pool.name} property. With the help of the Ibis server, 
this name is then used to locate other Ibis instances which belong to the
same pool. Since the Ibis server can service multiple pools simultaneously, 
each pool must have a unique name.

It is possible for pools to have a fixed size. In these so-called \emph{closed
world} pools, the number of Ibises in the pool is also needed to function 
correctly. This size must be set using the \texttt{ibis.pool.size} property. 
This property is only needed for closed-world pools.
When it is needed, but not provided, Ibis will throw an exception when it
is created.

\section{The ipl-run script}

To simplify running a Ibis application, a \texttt{ipl-run} script is
provided with the distribution. This script can be
used as follows

\begin{center}
\texttt{ipl-run} \emph{java-flags class parameters}
\end{center}

The script performs the first three steps needed to run an application
using Ibis. It adds the ipl.jar and all Ibis implementation jars to the
class path, and configures log4j. It then runs \texttt{java} with any
command line options given to it. Therefore, any additional options for 
Java, the main class and any application parameters must be provided as 
if \texttt{java} was called directly.

The \texttt{ipl-run} script needs the location of the Ibis
distribution. This must be provided using the IPL\_HOME environment
variable.

\section{Example}

To illustrate running an Ibis application we will use a simple "Hello
World" application. This application is started twice on a single
machine. One instance will send a small message to the other, which will
print it.

\subsection{Compiling the example}

The example applications for the Ibis communication library are
provided with the Ibis distribution, in the \texttt{examples} directory.
For convenience, these applications are already compiled. 

If you change any of the example, you will need to recompile them. This 
requires the build system \texttt{ant}\footnote{\url{http://ant.apache.org}}. 
Running \texttt{ant} in the examples directory compiles the examples.

Alternatively, they can be compiled using only \texttt{javac}. The sources are
located in the \texttt{src} directory of examples. Be sure to add
\texttt{ipl.jar} 
from the \texttt{lib} directory of the distribution to the class path.

\subsection{Running the example on Unix-like systems}

We will now run the example. All code below assumes that the IPL\_HOME
environment variable is set to the location of the Ibis distribution.

First, we will need a ipl-server. Start a shell and
run the \texttt{ipl-server} script:
\noindent
{\small
\begin{verbatim}
$ $IPL_HOME/scripts/ipl-server --events
\end{verbatim}
}
\noindent

By providing the \texttt{--events} option the server 
prints information on when Ibis instances join and leave the pool.

Next, we will start the application two times. One instance will act as the
"server", and one the "client". The application will determine who is who
automatically. Therefore we can using the same command line for both client 
and server. Run the following command in two different shells:

\noindent
{\small
\begin{verbatim}
$ CLASSPATH=$IPL_HOME/examples/lib/ipl-examples.jar \
    $IPL_HOME/scripts/ipl-run \
    -Dibis.server.address=localhost -Dibis.pool.name=test \
    ibis.ipl.examples.Hello
\end{verbatim}
}
\noindent

This sets the CLASSPATH environment variable to the jar file of the
application, and calls ipl-run. You should now have two running
instances of your application. One of them should print:

\noindent {\small \begin{verbatim} Server received: Hi there
\end{verbatim} } \noindent 

As said, the ipl-run script is only provided for convenience. To run
the application without ipl-run, the following command can be used:

\noindent
{\small
\begin{verbatim}
$ java \
    -cp \
    $IPL_HOME/lib/ipl-2.3.jar:$IPL_HOME/examples/lib/ipl-examples.jar \
    -Dibis.implementation.path=$IPL_HOME/lib \
    -Dibis.server.address=localhost \
    -Dibis.pool.name=test \
    -Dlog4j.configuration=file:$IPL_HOME/log4j.properties \
    ibis.ipl.examples.Hello
\end{verbatim}
}
\noindent

In this case, we use the \texttt{ibis.implementation.path} property to supply Ibis
with the jar files of the Ibis implementations. Alternatively, they
could also all be added to the class path.

\subsection{Running the example on Windows systems}

We will now run the example on a Windows XP system.
All code below assumes the IPL\_HOME
environment variable is set to the location of the Ibis distribution.

First, we will need a ipl-server. Start a command prompt window and
run the \texttt{ipl-server} script:
\noindent
{\small
\begin{verbatim}
C:\DOCUME~1\Temp> "%IPL_HOME%"\scripts\ipl-server --events
\end{verbatim}
}
\noindent

Note the quoting, which is needed when IPL\_HOME contains spaces.

By providing the \texttt{--events} option the server 
prints information on when Ibis instances join and leave the pool.

Next, we will start the application two times. One instance will act as the
"server", and one the "client". The application will determine who is who
automatically. Therefore we can using the same command line for both client 
and server. Run the following command in two different command prompt windows.
Again, note the absence as well as presence of quoting! Also, the ipl-run
command is split into multiple lines for readability. This should be just a single
line.

\noindent
{\small
\begin{verbatim}
C:\DOCUME~1\Temp> set CLASSPATH=%IPL_HOME%\examples\lib\ipl-examples.jar
C:\DOCUME~1\Temp> "%IPL_HOME%"\scripts\ipl-run
    "-Dibis.server.address=localhost" "-Dibis.pool.name=test"
    ibis.ipl.examples.Hello
\end{verbatim}
}
\noindent

This sets the CLASSPATH environment variable to the jar file of the
application, and calls ipl-run. You should now have two running
instances of your application. One of them should print:

\noindent {\small \begin{verbatim} Server received: Hi there
\end{verbatim} } \noindent 

As said, the ipl-run script is only provided for convenience. To run
the application without ipl-run, the following command can be used.
Again, the java command line is split into multiple lines for readability.
This should just be a single line.

\noindent
{\small
\begin{verbatim}
C:\DOCUME~1\Temp> java
    -cp "%IPL_HOME%"\lib\ipl-2.3.jar;"%IPL_HOME%"\examples\lib\ipl-examples.jar
    -Dibis.implementation.path="%IPL_HOME%"\lib
    -Dlog4j.configuration=file:"%IPL_HOME%"\log4j.properties
    -Dibis.server.address=localhost -Dibis.pool.name=test
    ibis.ipl.examples.Hello
\end{verbatim}
}
\noindent

In this case, we use the \texttt{ibis.implementation.path} property to supply Ibis
with the jar files of the Ibis implementations. Alternatively, they
could also all be added to the class path.

\section{Further Reading}

The Ibis web page \url{http://www.cs.vu.nl/ibis} lists all
the documentation and software available for Ibis, including papers, and
slides of presentations.

For detailed information on developing an Ibis application see the
Programmers Manual, available in the docs directory of the Ibis
distribution.

\end{document}
